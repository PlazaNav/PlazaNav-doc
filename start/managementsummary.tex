% Das Management Summary richtet sich in der Praxis an die "Chefs des Chefs", d.
% h. an die Vorgesetzten des Auftraggebers (diese sind in der Regel keine
% Fachspezialisten).
% Die Sprache soll knapp, klar und stark untergliedert sein.
% Zu verwenden ist folgenden Gliederung:
% - Ausgangslage - Vorgehen, Technologien - Ergebnisse - Ausblick (optional)

\chapter*{Management Summary}\addcontentsline{toc}{chapter}{Management Summary}

% TODO: 3 - 4 Seiten, 2 Bilder mit Outlook

% hier eine kurze Einleitung

\textbf{Ausgangslage}

\textbf{Vorgehen und Technologien}

\textbf{Ergebnisse}

\textbf{Ausblick}