% Der Abstract richtet sich an den Spezialisten auf dem entsprechenden Gebiet
% und beschreibt daher in erster Linie die (neuen, eigenen) Ergebnisse und
% Resultate der Arbeit. Es umfasst nie mehr als eine Seite, typisch sogar nur
% etwa 200 Worte (etwa 20 Zeilen). Es sind keine Bilder zu verwenden.

\chapter*{Abstract}\addcontentsline{toc}{chapter}{Abstract}

Die heutig gängigen Routing-Engines sind für den motorisierten Individualverkehr optimiert. Bei diesem halten sich die Verkehrsteilnehmer an vorgegebene Regeln und Strecken. Fussgänger allerdings optimieren intuitiv ihre Route, so wählen sie zum Beispiel über eine öffentliche Fläche den möglichst kürzesten Weg zum Ziel. Bestehende Routing-Engines navigieren entlang der Kante des Platzes, anstatt ihn direkt zu überqueren.

Es werden bestehende Algorithmen, Visibility-Graph und SpiderWeb-Graph, zur Traversierung von offenen Fussgänger-Flächen evaluiert, analysiert und optimiert. Mit einer Vorverarbeitung von OpenStreetMap-Daten wird gezeigt, wie eine Routing-Engine ein natürliches Fussgänger-Routing über offene Flächen unterstützen kann.

Als praktische Umsetzung wird in Python mit Hilfe des Services von search.ch ein Service für ein multimodales Routing mit öffentlichen Verkehrsmitteln erarbeitet. Mit einem eigens entwickelten Plugin für QGIS können die optimierten Routen visualisiert werden.

Durch die Vorverarbeitung von OpenStreetMap-Daten ist die Grundlage für ein natürliches Fussgänger-Routing geschaffen, auf welche gängige Routing-Engines aufsetzen können. In einem visuellen Vergleich zeigen die eingesetzen Algorithmen  deutlich bessere Ergebnisse als die bestehenden Implementationen in den Routing-Engines, während die Datenmenge für die Schweiz um weniger als 0.5\% steigt.

Mit dem entwickelten Backend und einem zugehörigen QGIS-Plugin können Benutzer mit einem beliebigen Start- und Endpunkt ein ÖV-Routing mit optimiertem Verhalten bei Fussgänger-Routen durchführen.

\cleardoublepage

\chapter*{Abstract}

Today's most notable open source routing engines are optimized for motorized traffic but show deficiencies in pedestrian routing. With open spaces, routers usually navigate along the edges of the available area, whereas pedestrians would naturally take shortcuts through the open space while avoiding obstacles.
Past research has shown multiple approaches to this problem. This project compares a few of the approaches used to address this problem.

Utilizing publicly available geographic data from OpenStreetMap and the help of existing algorithms, an implementation is proposed to optimize geographic data for existing routing engines that enhances the capacity to produce pedestrian routing approximating natural behavior. The optimization is refined using shortest-path algorithms to minimize additional data volume.

Furthermore, the optimized pedestrian routing is used in combination with existing services for public transport routing in Switzerland, providing a practical application addressing multimodal transportation. With a newly-developed plugin for QGIS, users are able to visualize the optimized routes.

The implementation of two of the different approaches toward data processing, visibility graph and SpiderWeb graph, demonstrates a clear improvement of routes for pedestrian navigation compared to existing methodology utilized in current routing engines. In the future, this implementation could serve as a reference to integrate these approaches directly into routing engines to enhance the usability of these programs for pedestrians and provide an optimized user experience regardless of the method of transit.