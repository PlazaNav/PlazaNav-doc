% Der Abstract richtet sich an den Spezialisten auf dem entsprechenden Gebiet
% und beschreibt daher in erster Linie die (neuen, eigenen) Ergebnisse und
% Resultate der Arbeit. Es umfasst nie mehr als eine Seite, typisch sogar nur
% etwa 200 Worte (etwa 20 Zeilen). Es sind keine Bilder zu verwenden.

\chapter*{Abstract}\addcontentsline{toc}{chapter}{Abstract}

Die heutig gängigen Routing-Engines sind für den motorisierten Individualverkehr optimiert. Bei diesem halten sich die Verkehrsteilnehmer an vorgegebene Regeln und Strecken. Fussgänger allerdings optimieren intuitiv ihre Route, so wählen sie zum Beispiel über eine öffentliche Fläche den möglichst kürzesten Weg zum Ziel. Bestehende Routing-Engines navigieren entlang der Kante des Platzes, anstatt ihn direkt zu überqueren.

Es werden bestehende Algorithmen zur Traversierung von offenen Fussgänger-Flächen evaluiert, analysiert und optimiert. Mit einer Vorverarbeitung von OpenStreetMap-Daten wird gezeigt, wie eine Routing-Engine ein natürliches Fussgänger-Routing durch offene Flächen unterstützen kann.

Als praktisches Beispiel wird mit Hilfe des Services von search.ch ein Prototyp für ein multimodales Routing mit öffentlichen Verkehrsmitteln erarbeitet. Mit einem eigens entwickelten Plugin für QGIS können die optimierten Routen visualisiert werden.

Durch die Vorverarbeitung von OpenStreetMap-Daten ist die Grundlage für ein natürliches Fussgänger-Routing geschaffen, auf welche gängige Routing-Engines aufsetzen können. Die eingesetzen Algorithmen zeigen deutlich bessere Ergebnisse als die bestehenden Implementationen in den Routing-Engines.

Mit dem entwickelten Prototypen und einem zugehörigen QGIS-Plugin können Benutzer mit einem beliebigen Start- und Endpunkt ein ÖV-Routing mit optimiertem Verhalten bei Fussgänger-Routen durchführen.

\cleardoublepage

Today's most notable open source routing engines are optimized for motorized traffic, but show deficiencies when it comes to pedestrian routing. With open spaces, routers usually navigate along the edges of the polygon, whereas pedestrians would naturally take shortcuts through the open space while avoiding any obstacles.
Past research has shown multiple approaches to this problem. In this thesis, some of these are approaches are compared and analyzed.

With publicly available geographic data from OpenStreetMap and the help of these algorithms, a proof-of-concept implementation is provided to prepare geographic data for an existing routing engine, with the aim of enhancing it with the capabilities to provide a natural behavior in pedestrian routing. The preparation step is optimized with shortest-path algorithms to keep the additional data volume as low as possible.

Furthermore, the optimized routing is used in combination with an existing service for public transport routing in Switzerland, providing a practical application with multimodal transportation. With a newly-developed plugin for QGIS, users are able to visualize the optimized routes.

The implementation of two different approaches --- visibility graph and SpiderWeb graph --- shows a clear improvement on routes for pedestrian navigation, compared to the methods that are used in current routing engines. In the future, this implementation could serve as a reference to integrate these approaches directly into routing engines.