
\section{Persönliche Berichte}
\label{sub:Persönliche Berichte}

\subsection{Robin Suter}
\label{Persönliche Berichte:Robin Suter}
Schon seit geraumer Zeit interessiere ich mich für Algorithmen für die Analyse von Graphen. Diese Arbeit bot mir eine gute Möglichkeit, dies auf ein praktisches Problem anwenden zu können. Die Struktur mit dem grossen theoretischen Teil gefiel mir gut, da es mir neben dem Software-Engineering auch eine Möglichkeit gab, mich wissenschaftlich mit einem Thema zu befassen.

Es war sehr motivierend, mit den \ac{OSM}-Daten zu arbeiten und damit die Vorverarbeitung zu implementieren. Dabei konnte ich auch schön beobachten, wie eine kleine Optimierung in der Komplexität einen sehr grossen Unterschied in der Laufzeit ausmachen kann.

Die Datenqualität der Kartendaten war eine Herausforderung, wir entdeckten immer wieder neue Grenzfälle, die in die Verarbeitung einbezogen werden mussten. Dies motivierte aber auch immer, unseren Code zu optimieren.

Die Zusammenarbeit im Team funktionierte stets sehr gut. Der gegenseitige Austausch war sehr hilfreich, um gemeinsam Problemlösungen zu finden und Ideen auszutauschen.

Insgesamt bin ich sehr zufrieden mit dieser Arbeit. Ich hoffe, dass unsere Software in Zukunft von der Open-Source-Community weiter verwendet werden kann, um das Routing zu verbessern.


\subsection{Jonas Matter}
\label{Persönliche Berichte:Jonas Matter}
Die Themengebiete Pfadoptimierung und \ac{GIS} begeistern mich seit einiger Zeit. Eine Arbeit in diesem Bereich schreiben zu dürfen, kam somit gelegen. PlazaRoute unterscheidet sich im Aufbau von meinen bisher durchgeführten Software-Projekten. Die Arbeit hat einen grossen theoretischen Fokus und war unter anderem geprägt von viel Wissensaufbau im Bereich der Flächenverarbeitungsalgorithmen und des Routings. 

Der Umgang mit grossen Datenmengen war eine Herausforderung, die spannend zu lösen war. Es war interessant zu sehen, was kleine Anpassungen für grosse Performanzgewinne mit sich bringen können.

Die Zusammenarbeit mit Robin Suter war wie auch in vorherigen Projekten eine Bereicherung. Die gegenseitigen Reviews und das Hinterfragen der Lösungen regen zum reflektieren an und haben zu einer hohen Qualität beigetragen, sei dies auf Seiten der Architektur, des Codes oder der Dokumentation.

Abschliessend kann ich sagen, dass ich mit dem Verlauf und dem Resultat der Arbeit sehr zufrieden bin und hoffe, dass PlazaRoute in der einen oder anderen Art dem Fussgänger zu gute kommen wird.