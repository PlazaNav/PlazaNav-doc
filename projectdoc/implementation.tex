
\section{Implementation}
\label{sec:Implementation}

\subsubsection{Komponenten}
\label{subsub:impl_Komponenten}

\paragraph{OSM Import}~\\
Die \acs{OSM}-Import Komponente liest das komplette für uns relevante Kartenmaterial (z.B. die Schweiz) als \ac{PBF} ein und sucht dabei nach Flächen, die wir bearbeiten wollen.

Dazu werden \emph{Osmium} und die dazugehörigen Python-Bindings \emph{pyOsmium} verwendet. \emph{Osmium} erkennt automatisch Flächen aus \ac{OSM} Multipolygone oder Relationen. Mit einem eigenen Handler können wir dabei gleich das Einlesen des Files auf die für uns interessanten Flächen beschränken, wie in Listing \ref{osmium_import_code} gezeigt wird.

\begin{listing}[ht]
    \inputminted{python}{projectdoc/listing/osmium_handler.py}
    \caption[Einlesen OSM-Daten mit Osmium]{Einlesen von OSM Daten mithilfe von \emph{Osmium}; Filterung auf für uns relevante Flächen}
    \label{osmium_import_code}
\end{listing}

\paragraph{OSM Verarbeitung}~\\
Die mit \emph{Osmium} importierten \ac{OSM} Daten sind noch reine \ac{OSM}-Objekte, auf denen keine Geometrie-Berechnungen angewendet werden können. Dazu wird die Python-Library \emph{Shapely} verwendet. \emph{Shapely} kann mit Geometrien umgehen und Algorithmen von \ac{GEOS} wie \code{intersection} und \code{contains} darauf anwenden.

Um die mit \emph{Osmium} importierten Objekte in \emph{Shapely} zu verwenden, werden diese ins \ac{WKB} Format übersetzt und \emph{Shapely} übergeben, wie in Listing \ref{shapely_import_code} gezeigt.

\begin{listing}[ht]
    \inputminted{python}{projectdoc/listing/shapely_import.py}
    \caption[Einlesen OSM Objekte in Shapely]{Übergabe von Osmium-Objekten zu Shapely für die Weiterverarbeitung}
    \label{shapely_import_code}
\end{listing}
