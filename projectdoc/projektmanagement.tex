
% Team, Rollen, Verantwortlichkeiten
% Aufwandschätzung, Zeitplan, Projektplan
\section{Projektmanagement}
\label{sec:Projektmanagement}

\subsection{Vorgehen}
\label{sub:Vorgehen}

Für die Studienarbeit wurde das agile Vorgehen SCRUM in Kombination mit \acs{RUP} gewählt. Grund für diese Entscheidung ist, dass das agile Vorgehen der noch zu Beginn unklaren Aufgabenstellung entgegenkommt, welche dann iterativ erarbeitet werden konnte. Die wöchentlichen Besprechungen und Reviews mit dem Betreuer ist ein weiterer Grund für diese Entscheidung. Die Kombination mit \acs{RUP} ermöglicht es, dass Projekt in einzelne Phasen aufzuteilen, um so das Ziel und die Zeit nicht aus den Augen zu verlieren.

\subsubsection{Entwicklung}
\label{sub:Entwicklung}

Der Source-Code, wie auch diese Arbeit wird mit Git verwaltet und ist auf Github abgelegt. Die Entwicklung und das Dokumentieren erfolgt nach dem Github-Flow. Der Master-Branch ist auf allen Repositories während der ganzen Zeit gesperrt, so dass er nur über Pull-Requests bearbeitet werden kann. Für jede User-Story wird ein Branch erstellt. Ist die User-Story implementiert, wird ein Pull-Request erstellt und dem anderen Projekt-Mitglied zum Review übergeben. Wird der Pull-Request akzeptiert, wird der Feature-Branch in den Master gemerged. Dieses Vorgehen hat den Vorteil, dass alle Änderungen, welche in den Master gelangen, ein Review durchlaufen müssen und so die Qualität hochgehalten werden kann.


\subsection{Zeitplanung}
\label{sub:Zeitplanung}

Die Arbeitspakte und Zeit wird mithilfe von Jira verwaltet. Für alle Tätigkeiten werden User Stories im Backlog erfasst, priorisiert und geschätzt. Die Schätzung der User Stories erfolgte mit Story Points. Die Arbeitszeitverbuchung wurde auf User Story-Stufe mit Stunden gemacht. 

\subsubsection{Phasen / Iterationen und Meilensteine}
\label{sub:Phasen / Iterationen und Meilensteine}

Die Studienarbeit wird in die \acs{RUP}-Phasen (Inception, Elaboration, Construction, Transition) aufgeteilt. Dabei wird jedoch eine von der gängigen Norm abweichende Aufteilung gewählt. Durch den theoretischen Fokus der Arbeit wird der Elaboration das grösste Zeitbudget zugeordnet. Dies ist auch der Grund warum mit einwöchigen Sprints gearbeitet wird.

\begin{table}[]
	\centering
	\caption{Phasen / Iterationen und Meilensteine}
	\begin{tabularx}{\textwidth}{lXXX}
		% Row 0
		\toprule
		\textbf{Sprint}        & \textbf{Sprint 0} & \textbf{Sprint 1} & \textbf{Sprint 2} \\ \midrule
		\textbf{Phase}         & Inception         & Inception         & Elaboration \\
		\textbf{Milestones}    & \textit{Aufgabenstellung}  & \textit{Aufgabenstellung}  & \textit{Stand Fussgänger-Routing über offene Flächen} \\
		\textbf{Zeitbudget}    & 1 Woche           & 1 Woche           & 1 Woche                                 \\
		\textbf{Arbeitspakete} & \nextitem Aufgabestellung-Brainstorming \nextitem JIRA aufsetzen \nextitem LATEX Doc aufsetzen \nextitem Vorgehen definieren              &  \nextitem Aufgabenstellung finalisieren \nextitem Travis aufsetzen \nextitem Python/Docker Know-How aufbauen \nextitem Einarbeitung QGIS            & \nextitem Visibility-Graph Know-How sammeln \nextitem Visibility-Graph QGIS Test \nextitem Spider-Web-Graph Know-How sammeln                                  \\ \bottomrule
		% Row 1
		\toprule
		\textbf{Sprint}        & \textbf{Sprint 3} & \textbf{Sprint 4} & \textbf{Sprint 5} \\ \midrule
		\textbf{Phase}         & Elaboration         & Elaboration         & Elaboration \\
		\textbf{Milestones}    & \textit{Stand Fussgänger-Routing über offene Flächen}  & \textit{Stand Fussgänger-Routing über offene Flächen}  & \textit{Stand Fussgänger-Routing über offene Flächen} \\
		\textbf{Zeitbudget}    & 1 Woche           & 1 Woche           & 1 Woche                                 \\
		\textbf{Arbeitspakete} & \nextitem Spider-Web-Graph QGIS Test \nextitem Skeleton-Graph Know-How sammeln \nextitem Skeleton-Graph QGIS Test              & TODO              & TODO                                    \\ \bottomrule
		% Row 2
		\toprule
		\textbf{Sprint}        & \textbf{Sprint 6} & \textbf{Sprint 7} & \textbf{Sprint 8} \\ \midrule
		\textbf{Phase}         & Elaboration         & Elaboration         & Elaboration \\
		\textbf{Milestones}    & \textit{Stand Fussgänger-Routing über offene Flächen}  & \textit{Stand Fussgänger-Routing über Strassen}  & \textit{Stand Fussgänger-Routing über Strassen} \\
		\textbf{Zeitbudget}    & 1 Woche           & 1 Woche           & 1 Woche                                 \\
		\textbf{Arbeitspakete} & TODO              & TODO              & TODO                                    \\ \bottomrule
		% Row 3
		\toprule
		\textbf{Sprint}        & \textbf{Sprint 9} & \textbf{Sprint 10} & \textbf{Sprint 11} \\ \midrule
		\textbf{Phase}         & Construction         & Construction         & Construction \\
		\textbf{Milestones}    & \textit{Prototyp Backend}  & \textit{Prototyp Backend}  & \textit{Prototyp Frontend} \\
		\textbf{Zeitbudget}    & 1 Woche           & 1 Woche           & 1 Woche                                 \\
		\textbf{Arbeitspakete} & TODO              & TODO              & TODO                                    \\ \bottomrule
		% Row 4
		\toprule
		\textbf{Sprint}        & \textbf{Sprint 12} & \textbf{Sprint 13} & \textbf{Sprint 13} \\ \midrule
		\textbf{Phase}         & Construction         & Transition         & Transition \\
		\textbf{Milestones}    & \textit{Prototyp Frontend}  & \textit{Abgabe SA}  & \textit{Abgabe SA} \\
		\textbf{Zeitbudget}    & 1 Woche           & 1 Woche           & 1 Woche                                 \\
		\textbf{Arbeitspakete} & TODO             & TODO              & TODO                                    \\ \bottomrule
	\end{tabularx}
\end{table}


\subsection{Risiken}
\label{sub:Risiken}

TODO