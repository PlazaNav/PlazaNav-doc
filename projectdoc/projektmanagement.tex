
% Team, Rollen, Verantwortlichkeiten
% Aufwandschätzung, Zeitplan, Projektplan
\section{Projektmanagement}
\label{sec:Projektmanagement}

\subsection{Vorgehen}
\label{sub:Vorgehen}

Für die Studienarbeit wurde das agile Vorgehen SCRUM in Kombination mit \acs{RUP} gewählt. Grund für diese Entscheidung ist, dass das agile Vorgehen der noch zu Beginn unklaren Aufgabenstellung entgegenkommt, welche dann iterativ erarbeitet werden konnte. Die wöchentlichen Besprechungen und Reviews mit dem Betreuer ist ein weiterer Grund für diese Entscheidung. Die Kombination mit \acs{RUP} ermöglicht es, dass Projekt in einzelne Phasen aufzuteilen, um so das Ziel und die Zeit nicht aus den Augen zu verlieren.

\subsubsection{Entwicklung}
\label{sub:Entwicklung}

Der Source-Code, wie auch diese Arbeit wird mit Git verwaltet und ist auf Github abgelegt. Die Entwicklung und das Dokumentieren erfolgt nach dem Github-Flow. Der Master-Branch ist auf allen Repositories während der ganzen Zeit gesperrt, so dass er nur über Pull-Requests bearbeitet werden kann. Für jede User-Story wird ein Branch erstellt. Ist die User-Story implementiert, wird ein Pull-Request erstellt und dem anderen Projekt-Mitglied zum Review übergeben. Wird der Pull-Request akzeptiert, wird der Feature-Branch in den Master gemerged. Dieses Vorgehen hat den Vorteil, dass alle Änderungen, welche in den Master gelangen, ein Review durchlaufen müssen und so die Qualität hochgehalten werden kann.


\subsection{Zeitplanung}
\label{sub:Zeitplanung}

Die Arbeitspakte und Zeit wird mithilfe von Jira verwaltet. Für alle Tätigkeiten werden User Stories im Backlog erfasst, priorisiert und geschätzt. Die Schätzung der User Stories erfolgte mit Story Points. Die Arbeitszeitverbuchung wurde auf User Story-Stufe mit Stunden gemacht. 

\subsubsection{Phasen / Iterationen}
\label{sub:Phasen / Iterationen}

Die Studienarbeit wurde in die gängigen \acs{RUP}-Phasen (Inception, Elaboration, Construction, Transition) aufgeteilt. 

TODO Grafik einbauen

\subsubsection{Meilensteine}
\label{sub:Meilensteine}

TODO

\subsection{Risiken}
\label{sub:Risiken}

TODO