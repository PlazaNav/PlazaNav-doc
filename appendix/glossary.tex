% Append the following to the XeLaTex command: |makeglossaries %

\newglossaryentry{BoundingBox}{
    name={Bouding Box},
    description={Ein Rechteck, welches durch zwei Längen- und Breitengrade definiert ist und nomalerweise dem Format \emph{min Longitude , min Latitude , max Longitude , max Latitude} folgt.}
}

\newglossaryentry{Einstiegspunkt}{
    name={Einstiegspunkt},
    description={Ein Punkt auf dem Rand eines äusseren Polygons einer Fussgängerfläche, der sich mit einer bestehenden Strasse oder mit einem Fussweg schneidet oder eine solche Linie berührt.},
    plural={Einstiegspunkte}
}

\newglossaryentry{Kante}{
    name={Kante}, % same explanation is used in the introduction chapter
    description={Eine ÖV-Haltestelle kann mehrere Plattformen umfassen. Normalerweise gehören zu einer ÖV-Haltestelle zwei Plattformen, je eine in jede Fahrtrichtung. Eine dieser Plattformen wird allgemein als Kante bezeichnet.},
    plural={Kanten}
}

\newglossaryentry{Shortest-Path}{
    name={Shortest-Path},
    description={}
}

\newglossaryentry{Routing-Engine}{
    name={Routing-Engine},
    description={}
}

\newglossaryentry{Geocoding}{
    name={Geocoding},
    description={}
}

\newglossaryentry{QGIS}{
    name={QGIS},
    description={}
}

\newglossaryentry{Plaza}{
    name={Plaza},
    description={Wird als Synonym für Fussgänger-Fläche verwendet.}
}

\newglossaryentry{Contraction Hierarchies}{
    name={Contraction Hierarchies},
    description={}
}

\newglossaryentry{Node}{
    name={OSM Node},
    description={}
}

\newglossaryentry{Way}{
    name={OSM Way},
    description={}
}
