% Append the following to the XeLaTex command: |makeglossaries %


\newglossaryentry{BoundingBox}{
    name={Bouding Box},
    description={Ein Rechteck, welches durch zwei Längen- und Breitengrade definiert ist und nomalerweise dem Format \emph{min Longitude , min Latitude , max Longitude , max Latitude} folgt. Im Kontext von Geometrien beschreibt eine \emph{minimale Bounding Box} das kleinstmögliche Rechteck, dass alle Punkte einer oder mehrerer Geometrien beinhaltet.}
}

\newglossaryentry{Einstiegspunkt}{
    name={Einstiegspunkt},
    description={Ein Punkt auf dem Rand eines äusseren Polygons einer Fussgängerfläche, der sich mit einer bestehenden Strasse oder mit einem Fussweg schneidet oder eine solche Linie berührt.},
    plural={Einstiegspunkte}
}

\newglossaryentry{Kante}{
    name={Kante}, % same explanation is used in the introduction chapter
    description={Eine ÖV-Haltestelle kann mehrere Plattformen umfassen. Normalerweise gehören zu einer ÖV-Haltestelle zwei Plattformen, je eine in jede Fahrtrichtung. Eine dieser Plattformen wird allgemein als Kante bezeichnet. Nicht zu verwechseln mit einer Kante eines Graphen, die eine Verbindung zwischen zwei Graph-Knoten beschreibt.},
    plural={Kanten}
}

\newglossaryentry{Shortest-Path}{
    name={Shortest-Path},
    description={Ein Begriff aus der Graphentheorie. Ein Shortest-Path beschreibt den kürzest möglichen Pfad zwischen zwei Knoten in einem gewichteten Graphen.}
}

\newglossaryentry{Routing-Engine}{
    name={Routing-Engine},
    description={Eine Software, die aus Kartendaten einen Graphen aufbereitet und Funktionalitäten anbietet, um auf diesen Routen zu berechnen.},
    plural={Routing-Engines}
}

\newglossaryentry{Geocoding}{
    name={Geocoding},
    description={Der Prozess, einer Postadresse eine Koordinate zuzuordnen. Der umgekehrte Weg, das Bestimmen einer Postadresse aus einer Koordinate, nennt sich \emph{Reverse Geocoding}.}
}

\newglossaryentry{QGIS}{
    name={QGIS},
    description={Ein frei verfügbares Geoinformationssystem für den Desktop für die Anzeige, Bearbeitung und Analyse von geografischen Daten.}
}

\newglossaryentry{Plaza}{
    name={Plaza},
    description={Wird in dieser Arbeit als Synonym für Fussgänger-Fläche verwendet.}
}

\newglossaryentry{Contraction Hierarchies}{
    name={Contraction Hierarchies},
    description={Ein Ansatz zur Optimierung von Routing, in dem der Routing-Graph hierarchisch erzeugt wird.}
}

\newglossaryentry{Node}{
    name={OSM Node},
    description={Ein geografischer Punkt im Modell von OpenStreetMap.},
    plural={Nodes}
}

\newglossaryentry{Way}{
    name={OSM Way},
    description={Eine Linie in OpenStreetMap, die gebildet wird, in dem mehrere Nodes verknüpft werden. Wenn ein Way einen geschlossenen Ring bildet, kann er auch eine Fläche beschreiben.},
    plural={Ways}
}

\newglossaryentry{OpenStreetMap}{
    name={OpenStreetMap},
    description={Ein Community-Projekt mit dem Ziel, eine frei verfügbare Karte der Erde zu erstellen, die von jedem bearbeitet und ergänzt werden kann.}
}

\newglossaryentry{Tag}{
    name={OSM Tag},
    description={Ein Schlüssel-Wert-Paar, das in OpenStreetMap einem Objekt zugewiesen wird, um dieses zu beschreiben. Tags können z.B. beschreiben, ob es sich bei einer Strasse um eine Haupt- oder Nebenstrasse handelt.},
    plural={Tags}
}