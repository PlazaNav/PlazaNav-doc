\section{Stand der Technik}
\label{sec:Stand der Technik}

In \ac{OSM} gibt es mehrere Arten, wie Mapper mit Flächen umgehen. So wirken einige Mapper dem Fussgängerrouting-Problem über Flächen mit zusätzlichen eingezeichneten Wegen entgegen oder die Fläche ist ganz befreit von Wegen.

In Abbildung \ref{fig:helvetiaplatz_comparison} und Abbildung \ref{fig:sechselaeutenplatz_comparison} ist ein Vergleich gängiger Routing-Engines aufgelistet, welche ein Fussgänger-Profil anbietet. In Abbildung \ref{fig:helvetiaplatz_comparison} sieht man schön, wie es bei \ac{OSM} möglich ist, dass die eingezeichneten Fusswege über denn Platz genutzt werden. Diese Route entspricht offensichtlich nicht einem natürlichen Fussgänger-Verhalten, da normalerweise der direkte Weg über den Platz gewählt wird, sofern keine Hindernisse im Weg sind. An dieser Stelle scheitert Google Maps. Der Vergleich von Google Maps und \ac{OSM} hinkt hier natürlich, da nicht beide auf den genau gleichen Datenbestand zurückgreifen. Betrachtet man in Abbildung \ref{fig:sechselaeutenplatz_comparison} hingegen den Sechseläutenplatz sieht man, dass ohne eingezeichnete Wege, alle getesteten Anbieter um den Platz herum führen.

\begin{figure}[ht]
\centering
\includegraphics[width=1\linewidth]{technicalreport/img/helvetiaplatz_comparison}
\caption[Fussgänger-Routing Vergleich]{Routing-Vergleich von verschiedenen Anbietern mit Fussgänger-Profil über den Helvetiaplatz, Zürich, Schweiz; Links: Google Maps, Mitte-Oben: openstreetmap.org mit GraphHopper, Mitte-Unten: openstreetmap.org mit Mapzen, Rechts: openrouteservice.org; Screenshots aufgenommen am 13.10.2017}
\label{fig:helvetiaplatz_comparison}
\end{figure}

\begin{figure}[ht]
\centering
\includegraphics[width=1\linewidth]{technicalreport/img/sechselaeutenplatz_comparison}
\caption[Fussgänger-Routing Vergleich]{Routing-Vergleich von verschiedenen Anbietern mit Fussgänger-Profil über den Sechseläutenplatz, Zürich, Schweiz; Links: Google Maps, Mitte-Oben: openstreetmap.org mit GraphHopper, Mitte-Unten: openstreetmap.org mit Mapzen, Rechts: openrouteservice.org; Screenshots aufgenommen am 13.10.2017}
\label{fig:sechselaeutenplatz_comparison}
\end{figure}


Abschliessend kann man sagen, dass alle getesteten Routing-Engines mit den Fussgänger-Profilen mit Stand 13.10.2017 scheitern, wenn keine Wege über die Fläche eingezeichnet sind und so die Dauer und Streckenlänge verfälscht wird.


\subsection{Bestehende Lösungsansätze}
\label{sub:Bestehende Lösungsansätze}


\subsubsection{Routing über offene Flächen}
\label{solution:Routing über offene Flächen}

Diese Arbeit nimmt Bezug auf zwei Lösungsansätze \cite{graser_visibility_graph}, \cite{dzafic_spider_web_graph}, welche sich aus verschiedenen Motivationsgründen mit dem Problem \ref{problem:Routing über offene Flächen} befassen. Diese werden in den folgenden Unterkapitel erläutert. Mit Pseudocode wird der Ansatz verdeutlicht. Beide Varianten werden in QGIS getestet. 

Zum Schluss wird ein Variantenvergleich durchgeführt und ein Fazit gezogen. Ziel ist es, einen Flächen-Routing-Algorithmus zu finden, welche für die Vorverabeitung der Flächen verwendet werden kann.  

\paragraph{Visibility Graph}~\\

TODO

\paragraph{Spider-Web Graph}~\\


Die Arbeit \cite{dzafic_spider_web_graph} befasst sich mit dem Flächenrouting für Nutzern von Elektrorollstühlen, indem ein Spinnennetz (siehe Abbildung \ref{fig:spiderweb}) über das Polygon gelegt wird. Dies hat den Vorteil, dass auf dem Polygon, sprich der Fläche zusätzliche Linien und Kanten vorhanden sind, wobei statische Hindernisse berücksichtigt werden, welche für das Routing verwendet werden können. Diese Idee wurde übernommen im folgenden übernommen.

\begin{figure}[ht]
\centering
\includegraphics[width=0.5\linewidth]{technicalreport/img/spiderweb}
\caption[Spinnennetz]{Spinnennetz}
\label{fig:spiderweb}
\end{figure}

Zur Verdeutlichung der Idee ist der Pseudocode \ref{Spiderweb Pseudocode} in zu betrachten.

\begin{listing}[ht]
    \inputminted{python}{technicalreport/listing/spiderweb_pseudocode.py}
    \caption{Spiderweb Pseudocode}
    \label{Spiderweb Pseudocode}
\end{listing}

\paragraph{Einschränkungen}~\\

TODO
% TODO mögliche Einschränkungen, welche die Varianten in ihrer Anwendung betreffen

\paragraph{Vergleich}~\\

TODO

% TODO Gegenüberstellung der Varianten, welche ist in welcher Situaiton sinnvoll

\paragraph{Fazit}~\\

TODO

% TODO Welche Variante wird verwendet

\subsection{Kurzbeschreibung und Charakterisierung}
\label{sub:Kurzbeschreibung und Charakterisierung}
TODO

\subsection{Defizite}
\label{sub:Defizite}
% Hinweise auf Weiterentwicklungs-, bzw. Verbesserungspotential
TODO
% hier Probleme, welche im Visiblity Graph Paper erwähnt sind aufgreifen.
