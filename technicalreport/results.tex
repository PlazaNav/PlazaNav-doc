\section{Resultate}
\label{sec:Resultate}

In diesem Kapitel werden die Resultate der Arbeit präsentiert. Die technische Beschreibung zur Implementation befindet sich in Teil \ref{chap:SW-Projektdokumentation} Kapitel \ref{sec:Implementation}.

\subsection{Zielerreichung}
\label{sub:Zielerreichung}

In der Evaluations-Phase wurde der Stand der Technik analysiert (Kap. \ref{sec:Stand der Technik}). Mit ersten Tests und Proof-of-Concept Implementationen in QGIS konnten dabei die Stärken und Schwächen der jeweiligen Algorithmen heraus gearbeitet werden. Es stellte sich schnell heraus, dass die beiden Ansätze \nameref{solution:Visibility-Graph} und \nameref{solution:SpiderWeb-Graph} sich am besten für unser Problem eignen. Parallel dazu wurden die Umsysteme analysiert, die für unsere Problemstellung eines multimodalen Routings benötigt werden (siehe Teil \ref{chap:SW-Projektdokumentation} Kap. \ref{sec:Analyse}). Dazu gehörte eine Analyse bestehender Routing-Engines und das Finden der ÖV-Haltestellen im näheren Umkreis.

Im nächsten Schritt wurden die Vorverarbeitung von \ac{OSM}-Daten mit der Flächenoptimierung implementiert. Die beiden Algorithmen --- \nameref{solution:Visibility-Graph} und \nameref{solution:SpiderWeb-Graph} --- wurden dabei parallel implementiert. Dies ermöglichte es uns, beide Ansätze optimal miteinander zu vergleichen (siehe Kapitel \ref{sec:Bewertung Routing über offene Flächen}).

Zusammen mit der Vorverarbeitung haben wir einen Service \cite{github:PlazaRoute} implementiert, der mit Hilfe der Routing-API von search.ch \cite{search_ch_route_api} ein Routing mit öffentlichen Verkehrsmitteln ermöglicht, wobei für das Fussgänger-Routing unsere optimierten Daten der Vorverarbeitung verwendet werden, um ein natürliches Fussgänger-Routing über offene Flächen zu erreichen. Dabei werden von einem beliebigen Startpunkt aus mehrere Haltestellen in der Umgebung gesucht und mit der Kombination von Fussgänger- und ÖV-Routing die optimale Route ermittelt.

Die Koordinaten der ÖV-Haltestellen werden dabei so optimiert, dass die ÖV-Haltestelle in die richtige Fahrtrichtung angesteuert werden kann.

Damit unser Service getestet und visualisiert werden kann, wurde ein Plugin für QGIS entwickelt \cite{github:PlazaRoute-qgis-plugin}. Dieses ermöglicht es, mit unserem Service Routen für den öffentlichen Verkehr interaktiv zu berechnen und visualisieren.

\subsection{Ausblick: Weiterentwicklung}
\label{sub:Ausblick: Weiterentwicklung}

Die von uns implementierte Vorverarbeitung für \ac{OSM}-Daten kann als Referenz dienen, um in Zukunft die Optimierung für Fussgänger-Flächen in bestehende Routing-Engines einzubauen. Es wäre sinnvoll, die Algorithmen direkt in Routing-Engines zu integrieren, statt in einem sepparaten Schritt zuerst \ac{OSM}-Daten aufbereiten zu müssen.

Die jetzige Lösung bietet noch weiteren Raum für Optimierung. So können im Moment einige Plätze nicht verarbeitet werden, weil zu wenig \glspl{Einstiegspunkt} existieren, obwohl in der Realität der Platz problemlos begehbar wäre. Ein ähnliches Problem besteht auch, wenn mehrere Fussgänger-Flächen direkt aneinander liegen. Ansätze für Lösungen dazu werden in den Kapitel \ref{subsub:Verbesserung_Einstiegspunkte} respektive \ref{subsub:Routing bei zwei benachbarten Flächen} diskutiert.

Die jetzige Lösung mit Python stösst mit der Performanz an ihre Grenzen. Es ist denkbar, eine Lösung mit PostGIS oder C++ zu realisieren.

\subsection{Persönliche Berichte}
\label{sub:Persönliche Berichte}

\subsubsection{Robin Suter}
\label{Persönliche Berichte:Robin Suter}
TODO

\subsubsection{Jonas Matter}
\label{Persönliche Berichte:Jonas Matter}
Die Themengebiete Pfadoptimierung und GIS begeistern mich seit einiger Zeit. Eine Arbeit in diesem Bereich schreiben zu dürfen, kam somit gelegen. PlazaRoute unterscheidet sich im Aufbau von meinen bisher durchgeführten Software-Projekten. Die Arbeit hat einen grossen theoretischen Fokus und war unter anderem geprägt von viel Wissensaufbau im Bereich der Flächenverarbeitungsalgorithmen und des Routings. 

Der Umgang mit grossen Datenmengen war eine Herausforderung, die spannend zu lösen war. Es war interessant zu sehen, was kleine Anpassungen für grosse Performanzgewinne mit sich bringen können.

Die Zusammenarbeit mit Robin Suter war wie auch in vorherigen Projekten eine Bereicherung. Die gegenseitigen Reviews und das Hinterfragen der Lösungen regen zum reflektieren an und haben zu einer hohen Qualität beigetragen, sei dies auf Seiten der Architektur, des Codes oder der Dokumentation.

Abschliessend kann ich sagen, dass ich mit dem Verlauf und dem Resultat der Arbeit sehr zufrieden bin und hoffe, dass PlazaRoute in der einen oder anderen Art dem Fussgänger zu gute kommen wird.


\subsection{Dank}
\label{sub:Dank}

Wir möchten folgenden Personen für ihre Unterstützung und Mitwirkung bei dieser Arbeit danken.

\textbf{Prof. Stefan Keller, IFS Institut für Software,} für die Zeit, Ressourcen, Kontakte, Know-How und Unterstützung, von welcher wir jederzeit profitieren konnten.

\textbf{Christian Helbling, localsearch,} für den Erfahrungsaustausch im Bereich von Fahrplandaten und search.ch.

\textbf{Mitarbeiter, IFS Institut für Software,} für den regen Know-How-Austausch und die Unterstützung bei der Produktivsetzung.
