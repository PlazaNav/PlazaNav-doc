\section{Umsetzungskonzept}
\label{sec:Umsetzungskonzept}
% z.T. Wiederholung im Groben, z.T. Verweise auf Teil II-Kapitel

Die Arbeit besticht durch einen grossen theoretischen Teil, welchem zu Beginn grosser Bedeutung beigemessen wird, da der Rest der Arbeit massgeblich von den Ergebnissen dieses Teils abhängt. So werden zuerst Algorithmen evaluiert und mit Tests in QGIS auf ihre Machbarkeit geprüft, um eine erste Bewertung der Algorithmen durchführen zu können. Dadurch ist klar, welche sich für die Implementierung eignen und welche verworfen werden können.

Parallel werden Abklärungen bezüglich der Fremdsysteme durchgeführt. Nach der Elaboration-Phase ist klar, auf welche Algorithmen gesetzt werden kann und ob die Fremdsysteme die gewünschten Anforderungen erfüllen und verwendet werden können.

Mit diesem Wissen lässt sich die Vorverarbeitung der \ac{OSM}-Daten und das Backend, welches ein multimodales Routing anbietet, umsetzen.

Steht das Backend und kann eine Routing-Engine auf die von uns vorverabeitenden \ac{OSM}-Daten operieren, wird die Visualisierung der Ergebnisse in einem QGIS-Plugin sichtbar gemacht.
